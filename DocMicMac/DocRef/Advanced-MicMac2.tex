\chapter{Advanced matching, practical aspect}


\section{Cost function}


The discrete version of the cost function is:


\begin{equation}
   E(Z) = \sum Corr(i,j,Z(i,j)) + F(|Z(i+1,j)-Z(i,j)|)  + \dots
\end{equation}

The $F$ function allows to control the a priori on the Z:

\begin{itemize}
   \item  if the desired model is smooth, a convex   $F$  can be adequate (it's better to
          climb a given jump by regular step);

   \item  if  the desired model has many discontinuities, a $F$   concave can be adequate
          (it's better to climb a given jump in one single step);

   \item when there is no strong a prior, the  default choice is to have $F$ linear;

\end{itemize}


The basic form of $F$ has two parameters $R$ and $Q$ :

\begin{equation}
   F(\Delta_Z) =  R |\Delta_Z| + Q {\Delta_Z}^2
\end{equation}


This allows to create linear and convex function. To create concavity,
there exists parameters $S$ and $A$ such that:

\begin{equation}
   F(\Delta_Z) =  R |\Delta_Z| + Q {\Delta_Z}^2 ,  |\Delta_Z| < S
\end{equation}

\begin{equation}
   F(\Delta_Z) =  R S +   R A (|\Delta_Z|-S)   + Q {\Delta_Z}^2 ,  |\Delta_Z| \geq  S
\end{equation}


Typically this means that when $Z$ is over the threshold $S$, the slope is multiplied by $A$.

\begin{itemize}
   \item {\bf ZRegul} add a linear term $ZRegul |\Delta_Z| $ to $F$;
\end{itemize}


The MicMac tag for this parameters are:

\begin{itemize}
   \item {\tt <ZRegul>}  for $R$;
   \item {\tt <ZRegul\_Quad>}  for $Q$;
   \item {\tt <SeuilAttenZRegul>}  for $S$;
   \item {\tt <AttenRelatifSeuilZ>}  for $A$;
\end{itemize}



%=================================================================

\section{Exporting the score}

\subsection{Default behaviour}

{\tt Malt}  generate an image which for each pixel
$i,j$ contains   $Corr(i,j,Z_{Opt}(i,j))$  where $Z_{Opt}(i,j))$
is the solution computed by {\tt Malt}.

Internally these value are normlized between $-1$ and $1$ 
\footnote{which is "natural" when it is the centered correlation coefficient}
and they  are resampled between $0$ and $255$. Visualized as black and white
images, the white correspond to "good" matches.

The name of the file containing this correlation maps are Correl\_STD-MALT\_Num\_XXX.tif
In MicMac , it is possible to generate or not this images with the tag {\tt <GenImagesCorrel>}.

\subsection{Exporting the correlation cube}

Sometime the user may  want to know $Corr(i,j,k)$ not only for $k=,Z^{Opt}(i,j)$ for all 
the computed value. This is possible using {\tt GCC} with {\tt Malt} and {\tt GenCubeCorrel}
in {\tt MicMac}. The structure of the data is a bit more comple as, due to the multi
scale approach of {\tt MicMac}, $Corr(i,j,k)$ is not computed  for  all $k$ but only
for $k \in [Z^{Min}(i,j),Z^{Max}(i,j)]$; also due to the parallelization in independant
process, the export is done independantly for each tile. The structure is the following :

\begin{itemize}
    \item  for each step $k$ of the macthing a folder {\tt Cubek} is created;
    \item  for each tile  :
    \begin{itemize}
          \item   let {\tt X,Y}  be the origin of the tile (pixel $(0,0)$ of this tile will correspond to {\tt X,Y} of the  the global data);
          \item   two files  {\tt Data\_X\_Y\_ZMin.tif} and {\tt Data\_X\_Y\_ZMax.tif}
                  are created, these files are $16$ bits signed tiff images, and contain the $Z^{Min}(i,j)$ and $Z^{Max}(i,j)$;
          \item    a file   {\tt  Data\_X\_Y\_Cube.dat} is created, it is a raw file that contains on after each other
                   the $Corr(i,j,k)$, normalized between $0$ and $255$ , and store on $8$ bits unsigned int;
                    the order of storage is  :
                   \begin{itemize}
                         \item     $Corr(0,0,Z^{Min}(0,0))$   $Corr(0,0,Z^{Min}(0,0)+1)$  \dots $Corr(0,0,Z^{Max}(0,0)-1)$
                         \item     $Corr(1,0,Z^{Min}(1,0))$   $Corr(1,0,Z^{Min}(1,0)+1)$  \dots $Corr(1,0,Z^{Max}(1,0)-1)$
                         \item     \dots
                   \end{itemize}
    \end{itemize}
\end{itemize}




For user interested, the code generating the data  can be found in 
{\tt src/uti\_phgrm/MICMAC/cSurfaceOptimiseur.cpp},
looking for the variable {\tt mCubeCorrel}.










