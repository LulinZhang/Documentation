\documentclass[11pt,a4paper,oneside]{book}
\usepackage[utf8]{inputenc}
\usepackage[english]{babel}
\usepackage{amsmath}
\usepackage{amsfonts}
\usepackage{amssymb}
\usepackage{graphicx}
\usepackage{lmodern}
\usepackage{lscape}
\usepackage{caption}
\usepackage{cleveref}
%
\author{l'équipe de MicMac}
\title{Traitement photogrammétrique des acquisitions linéaires avec l'outil MicMac}


\makeatletter
\newcommand\footnoteref[1]{\protected@xdef\@thefnmark{\ref{#1}}\@footnotemark}
\makeatother

%%%%%%%%
\begin{document}
\maketitle
\tableofcontents
%%%%%%%%%%%%%%%%%%%%%%%% 
\chapter{Desciption of the dataset}
generally and the dataset processed here

%%%%%%%%%%%%%%%%%%%%%%%% 
\chapter{Estimation des poses}

\section{Orientation interne et relatif -- auto-calibration}\label{sec:oriRel}
Pour la calibration d'une acquisition linéaire avec peu de points de contrôle on utilise les modèles de distorsion unifié (\textit{Unified polynomial model} caractérisé dans le xml par {\tt <ModUnif>}). 
La distorsion finale d'une camera sera une composition de plusieurs modèles de distorsion (par example une composition des polynômes radiaux et génériques). A la sortie du traitement ils ont les paramètres des cameras (selon le modèle de calibration choisi) et les paramètres d'orientation relatif. Le traitement se déroule dans 3 étapes:
%
\begin{enumerate}
\item \textbf{pre-calibration}, sur un sous-ensemble d'images, où seulement la focal $f$ et le point principal $pp$ sont considérés comme les paramètres libres;
\item \textbf{calibration avec l'approche standard étendu} qui a pour but corriger davantage erreurs causés par les imprécisions de la lentille et du capteur; cette fois-ci les erreurs sont {de la nature physique} (\underline{l’approche physique}) et on le corrige avec des polynômes radiaux, decentriques ainsi qu'avec une fonction affine; sur un sous-ensemble d'images ou l'ensemble entier; les paramètres libres sont:
	\begin{itemize}
	\item la focal $f$,
	\item le points principal $pp$,
	\item trois paramètres de la distorsion radiale symétrique ($R^3$, $R^5$, $R^7$),
	\item (optionnel) quatre paramètres de la distorsion asymétrique (decentrique) (polynôme du degré 2)
	\item deux paramètres de la distorsion affine 
	\end{itemize}
\item \textbf{calibration finale très fine} a pour but corriger les erreurs résiduelles non-radial avec les polynômes génériques du haute degré, sans se poser des questions sur la nature des erreurs (\underline{l’approche phénoménologique}); à ce stade, la calibration obtenu dans l'étape précédent est considère constant alors que en cours du traitement les coefficients du polynôme générique vont être estimés.
\end{enumerate}


%\subsection{Pre-calibration}
%\subsection{Calibration avec l'approche standard étendu}
%\subsection{Calibration finale très fine (approche étendu à corriger les erreurs résiduelles de nature non-radial)}

\section{Paramétrage dans MicMac}
%
Dans MicMac il y a quatre fonctions qui gèrent toutes les transformation de la calibration ($R2 \rightarrow R2$) appliqués dans le plan focal (l'image) :
\begin{itemize}
\item les polynômes radiaux (en utilisant le modèle {\tt FourXXx2} le degré est modifié par le DegRadMax) (dans DocMicMac Eq. 14.11 et A.15 )
\item les polynômes decentriques (en utilisant le modèle {\tt FourXXx2} le degré est modifié par le DegGen) (dans DocMicMac Eq. A23, A.24) 
\item les polynômes générique (dans le modèle {\tt AddPolyX})
\item la fonction affine (2 termes liberé si DegGen=\{1,2\}, dans DocMicMac Eq. A.10)
\end{itemize}
%
\subsection{Pre-calibration et calibration étendu}
%
Le modèle utilisé dans les deux premiers étapes est du type {\tt <ModUnif>}\footnote{voir le tag {\tt <CalibDistortion>} dans {\tt ParamChantierPhotogram.xml} pour apprendre d'autres types}, appelé depuis la ligne de commande par {\tt mm3d Tapas} avec un de modèels suivantes: {\tt Four7x2, Four15x2, Four19x2}. Le Tableau~\ref{tab:fourXX} contient une récapitulatif sur les differentes modèles et leur paramètres. Vous voyez que en jouant avec les arguments {\tt DegRadMax} et {\tt DegGen} on arrive aux modèles classiques comme {\tt FraserBasic}, {\tt RadialBasic} ou {\tt  RadialExtended}. Pre-calibration et calibration étendu modélise les erreurs d'un capteur avec les polynômes radiaux, decentriques, et la fonction affine.
%
\paragraph{Paramétrage dans les fichiers xml}
%
Inside the calibration XML file, the unified calibration model {\tt <ModUnif>} for {\tt FourXXx2} models holds the  {\tt eModeleRadFourXXx2} enumerated values\footnote{\label{fn:unif}This sub-type is important as the significance of the parameter value depends on it.}. Table~\ref{tab:XML_fourXX} explains the parameter encoding for the  {\tt FourXXx2} calibration model.\\
%\newpage 
%
\\
\begin{tabular} { | c | c | }  
 \hline \hline
{\tt Params[0]} &   Terme affine \\ \hline
{\tt Params[1]} & Terme affine \\ \hline
{\tt Params[2]} &  Polynôme decentrique terme quadratique (DegGen=2) \\ \hline
{\tt Params[3]} &  Polynôme decentrique terme quadratique (DegGen=2) \\ \hline
{\tt Params[4]} &   polynôme decentrique terme quadratique (DegGen=2) \\ \hline
{\tt Params[5]} &  Polynôme decentrique terme quadratique (DegGen=2) \\ \hline
{\tt Params[6]} &  Centre de distortion en x \\ \hline
{\tt Params[7]} &   Centre de distortion en y \\ \hline
{\tt Params[8+]} &   Polynômes radiaux ($R^3, R^5, R^7,$ … ) \\ \hline \hline
  

\end{tabular}\captionof{table}{Paramétrage du fichier xml d'une modèle {\tt FourXXx2}.}\label{tab:XML_fourXX}
%
\subsection{Calibration finale très fine}
%
Dans l'étape dernier on utilise le modèle {\tt AddPolyX} qui est également du type {\tt <ModUnif>}. En le faisant, MicMac garde le paramètres internes d'entre (calculé dans le pas précédent) figées, et ajoute une transformation en plus. Ainsi, toutes le pixels dans l'image subi un double décalage. La deuxième transformation est un polynôme générique et ses paramètres sont inconnus ainsi ils sont élus en cours de traitement.
\\
\\
La nomenclature dans MicMac e.g. {\tt mm3d Tapas ...}\\
{\tt AddPolyDeg2 (=AddPolyDeg2 DegGen=2)} – polynôme générique de degré 2\\ 
{\tt AddPolyDeg3 (=AddPolyDeg2 DegGen=3)} – polynôme générique de degré 3 
%
\paragraph{Paramétrage dans les fichiers xml}
%
Inside the calibration XML file, the unified calibration model {\tt <ModUnif>} for {\tt AddPolyDegX} models holds the  {\tt eModelePolyDegX} enumerated values\footnoteref{fn:unif}. Les valeurs respectives dans {\tt Param[]} correspond aux coefficients d'une polynôme générique, comme celles dans les Eqs.~\eqref{eq:polydeg2} and \eqref{eq:polydeg3}.


\begin{equation}
   P_2 \begin{pmatrix}  x \\  y  \end{pmatrix}
   = \begin{pmatrix}  x \\  y  \end{pmatrix}
     + \begin{pmatrix}  p_1 x  + p_2 y -2 p_3 x^2  + p_4 xy + p_5 y^2
                     \\ -p_1y + p_2 x  + p_3 xy - 2 p_4 y^2 + p_6 x^2 \end{pmatrix} \label{eq:polydeg2}
\end{equation}


\begin{equation}
   P_3 \begin{pmatrix}  x \\  y  \end{pmatrix}
   = P_2 \begin{pmatrix}  x \\  y  \end{pmatrix}
     +    \begin{pmatrix}  p_7 x^3 + p_8 x^2y + p_9 xy^2 + p_{10} y^3
                        \\  p_{11} x^3 + p_{12} x^2y + p_{13} xy^2 + p_{14} y^3   \end{pmatrix} \label{eq:polydeg3}
\end{equation}


%
\begin{landscape}
\begin{tabular} { | c | c | c | c| c| c |}  
 \hline
{\bf Polynômes  }        &  {Radial polynôme  } & {Center of distortion  } & {Decentrique polynôme} &  {Affine terme } &  {Somm}  \\ 
{ Unified distortion model } & { ($R^3$, $R^5$ … )} & {($Cx, Cy$)} & {(A25-30 ; DegGen=2)} &  { A.10)} &  { }  \\  \hline \hline

{\tt Four19x2} & 11 & 2 & 4 & 2 & 19 \\  \hline

{\tt Four15x2} & 7 & 2 & 0  & 2  & 11 \\ \hline

{\tt Four15x2}  DegGen=1 & 7 & 2 & 0 & 2 & 11  \\ \hline
 

{\tt Four15x2} DegGen=2 & 7 &  2 & 4 & 2 & 15  \\ \hline
 

{\tt Four15x2} & 0 & 0 & 0 & 2 & 2 \\
DegRadMax=0 & & & & & \\
DegGen=1  & & & & & \\ \hline

{\tt Four15x2} & 0 & 0 & 0 & 0 & 0 \\
DegRadMax=0 & & & & & \\
DegGen=0 & & & & & \\ \hline

{\tt Four15x2} & 0 & 0 & 4 & 2 & 6 \\
DegRadMax=0 & & & & & \\
DegGen=2 & & & & & \\ \hline

{\tt Four15x2} & 1 & 2 & 4 & 2 & 9 \\
DegRadMax=1 & & & & & \\
DegGen=2 & & & & & \\ \hline


{\tt Four15x2 } & 3 & 2 & 0 & 2 & 7 \\  
DegRadMax=3 & & & & & \\ \hline


{\tt Four15x2} & 3 & 2 & 0 & 0 & 5 \\
DegRadMax=3 & & & & & \\
DegGen=0 & & & & & \\ \hline \hline

\end{tabular}\captionof{table}{Une liste (pas exhaustif) de modèles étendu de camera et leur parameters.}\label{tab:fourXX}
\end{landscape}
%
\section{Orientation absolut -- geo-référencement}
%
At this point, the camera calibration is known, and so is the relative orientation of a set of images. There exist two ways to transform the results from a relative coordinate system to an absolute/reference coordinate system:
\begin{itemize}
\item via a spatial similarity transformation with {\tt GCPBascule}\footnote{It is supposed that there is no on-board GNSS/IMU data available} (with 7 or more parameters if non-linear effects are to be compensated for);
\item via a bundle adjustement with {\tt Campari} (the number of parameters is proportional to the number of cameras as by default for each camera both its pose and interior parameters are re-stimated); {\tt Campari} is a least squares refinement step requiring initial value for all its unknowns thus it is always preceded by {\tt GCPBascule} (or other {\tt Bascule}-like tool) .
\end{itemize}
%
\subsection{The bending effect}
In long and linear acquisitions from a nadiral point of view, the reconstructed camera poses are prone to be biased. The bias manifests in camera poses following a bent trajectory. This effect is due to the imaging configuration that is suboptimal for camera self-calibration (in french auto-calibration). To overcome this effect, {\tt MicMac} proposes two strategies:
\begin{itemize}
\item a less rigorous one and \underline{with few control points}: the so-called non-linear Bascule where  to compensate for the bending artefact\footnote{It is present in the camera "trajectory" and the 3D structure.}, on top of the 7 parameters some polynomials are defined and estimated (see Subsection~\ref{subsec:basc});
\item a more rigorous one but \underline{with abundant control points}: the bundle adjustment as explained in Subsection~\ref{subsec:compensation}
\end{itemize} 
%
\subsection{La transformation de similitude (Bascule)}\label{subsec:basc}
%
Given at least 3 control points (GCP) visible in at least two images the {\tt GCPBascule} tool, by default, performs a 7-parameter spatial similarity transformation (3 rotations, 3 translations and a scale factor). Most favourably the three GCP shall be distributed across the site, rather than located one next to another.\par 
%
To remove the bending effect from the data the {\tt GCPBascule} allows to define some additional correction funtions, namely some 3D generic polynomials. During the estimation phase, it is imposed that the polynomials minimize the differences between the GCPs furnished by external measurements, and those calculated from the bundles' intersection. \par 
%
{\tt MicMac} lets the user define a generic polynomial function for each coordinate, i.e.:
%
\begin{itemize}
   \item[--] $C(X',Y',Z') = (X'^c,Y'^c,Z'^c)$;
   \item[--] $X'^c = \sum c^x_{ij} X'^i Y'^j$
   \item[--] $Y'^c = \sum c^y_{ij} X'^i Y'^j$
   \item[--] $Z'^c = \sum c^z_{ij} X'^i Y'^j$
\end{itemize}
%
For example, suppose that (i) the acquisition is made from a single strip, (ii) the errors are only along the $Z^c$-coordinate,  and (iii) they depend only on  the "main" variable $X'$, a possible model could be :
%
\begin{itemize}
   \item[--] $X'^c = 0$
   \item[--] $Y'^c = 0$
   \item[--] $Z'^c =  c^z_{00} + c^z_{10} X' + c^z_{20} X'^2$
\end{itemize}
%
\subsection{Compensation par ajustement de faisceaux}\label{subsec:compensation}
%
Unlike in the {\tt Tapas }, where only the tie points can serve as the input observations, the {\tt Campari} tool permits to include heterogeneous observations in the adjustment routines. The additional observations in the majority of cases are the ground control points or the positions of the camera's optical center measured by some GNSS system. In other words, these are pieces of information that handle the georeferencing of your data.\par 
%
Bundle adjustment "by compensation" is a refinement stage and it expects that initial values of all the unknowns are provided. Consequently, it required that the relative orientation calculated in Section~\ref{sec:oriRel} is transformed to the reference coordinate system prior to the compensation. All the tools containing the {\tt Bascule} keyword can execute the transformation, yet in the considered scenario we would use the {\tt GCPBascule}.\par 
%
Once again, the difference between {\tt GCPBascule} and {\tt Campari} is in the mathematical model. {\tt GCPBascule} estimates the parameters of the spatial similarity transformation while {\tt Campari} works with the collinearity equations.

%
\section{Méthodes d’évaluation de résultats}
%
There exist \textit{internal} and \textit{external} accuracy measures. The internal accuracy refers to how well the mathematical model explains the observations. Examples of internal checks in Mic 

 
%after Relative              - residuals, MMTestOrient
%after GCPBascule standard   - residuals on GCP + visual in AperiCloud
%after GCPBascule non-linear - residuals on GCP + APeriCloud
%after Campari				- residuals on check points 

%
\section{Traitement dans MicMac}



\subsubsection{Les commandes}

%%%%%%%%%%%%%%%%%%%%%%%% 
\chapter{Mise en correspondance et production d'une orthomosaic}

\section{Les géométries du traitement}
\section{L’algorithme d'appariement dense et ses paramètres}
\section{Traitement dans MicMac}
\end{document}