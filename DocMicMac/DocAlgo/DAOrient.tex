\chapter{Algorithm on orientation}

    % - - - - - - - - - - - - - - - - - - - - - - - - - - - - - - - -
    % - - - - - - - - - - - - - - - - - - - - - - - - - - - - - - - -
    % - - - - - - - - - - - - - - - - - - - - - - - - - - - - - - - -

\section{Tomasi-Kanabe}

This section gives a brief description of the Tomasi-Kanabe algorithm
that is (will be) used in Martini as a suplementary test of
orientation for long focal camera. See~\cite{TomKan} , the original paper,
for more details.


Let $C_1,C_2, \dots C_N$ be $N$ camera with infinite focal lentgh, classically
the projection fonction are ortho centric projection, and there is a strict redundancy between principal-point/origin and  focal/origin. The projection fonction $\pi_n$ on camera $C_n$ is given by :

\begin{equation}
   \pi_n(P) =  (A_n,B_n)   + S_n * (i_n . P , j_n.P) \label{TomKanProj}
\end{equation}

Where :

\begin{itemize}
   \item  $i_n$ and $j_n$ are the axes of the camera they are the two first vector
          of an othonormal repair $||i_n|| = ||j_n|| = 1$ , $i_n . j_n = 0 $ ,
          we note $k_n = i_n \wedge j_n$.
   \item  the translation $(A_n,B_n)$ represent the "principal point/origin" on $i_n,j_n$;
   \item  the scaling factor $S_n$ represent the "focal length/origin" on $k_n$.
\end{itemize}


Having $M$ point seen in the $N$ camera  (we know the projection of these 
point but not their $3d$ coordinates in some common ground space) we note :

\begin{itemize}
    \item $P_m , m \in[1,M]$  the unknown $3d$ coordinates of the $m^{th}$ points;
    \item $(u_{n,m},v_{n,m}), m \in[1,M], n \in [1,N] $   the $2d$ coordinates of projection
          of $P_m$ points on $C_n$ ;
    \item we have :
\end{itemize}

\begin{equation}
    \pi_n(P_m) = (u_{n,m},v_{n,m})
\end{equation}
As the repair in which we want to express $P_k$ is arbirtrary, we decide to  set the 
origin at the center of mass; we then have   :


\begin{equation}
   \sum_{m=1}^{M} P_m = 0 \label{TomKanCDM}
\end{equation}

As equation~\ref{TomKanProj} is linear : 


\begin{equation}
     \forall n : \sum_{m=1}^{M} \pi_n(P_m)  
     =  (A_n,B_n) * M + S *(i_n. \sum_{m=1}^{M} P_m,j_n. \sum_{m=1}^{M} P_m)
     = (A_n,B_n) * M
     = \sum_{m=1}^{M} (u_{n,m},v_{n,m})
\end{equation}

So we see that $A_n,B_n$ can easily be computed by  :


\begin{equation}
   A_n = \frac{\sum_{m=1}^{M}u_{n,m}}{M} ;
   B_n = \frac{\sum_{m=1}^{M}v_{n,m}}{M} ; \label{TomKanAB}
\end{equation}

As in~\cite{TomKan}, to ligthen notation, we suppose we begin by normalise $u_{n,m}$ and $v_{n,m}$ 
by substracting the center of mass, and equation~\ref{TomKanAB} resume to $A_n=B_n=0$ and
equation~\ref{TomKanProj} can be writen :


\begin{equation}
   (u_{n,m},v_{n,m}) =  \pi_n(P_m) =   S_n * (i_n . P_m, j_n.P_m) \label{TomKanProj2}
\end{equation}


We note $M^u_v$ the matrix :


\begin{equation}
M^u_v=
\left( \begin{array}{cccc} 
             u_{1,1} & u_{2,1}  & \dots & u_{N,1} \\ 
             u_{1,2} & u_{2,2}  & \dots & u_{N,2} \\
              \dots &  \dots  & \dots &   \dots   \\
             u_{1,M} & u_{2,M}  & \dots & u_{N,M} \\
             v_{1,1} & v_{2,1}  & \dots & v_{N,1} \\ 
              \dots &  \dots  & \dots &   \dots   \\
             v_{1,M} & v_{2,M}  & \dots & v_{N,M} 
        \end{array} 
\right)
\label{TomKanMuv}
\end{equation}

Using equation~\ref{TomKanProj2}, equation~\ref{TomKanMuv} can be writen :


\begin{equation}
   M^u_v= \left( \begin{array}{ccc} 
             S_1*i^x_1 & S_1*i^y_1 & S_1*i^z_1 \\
             S_2*i^x_2 & S_2*i^y_2 & S_2*i^z_2 \\
             \dots & \dots & \dots \\
             S_M*i^x_M & S_M*i^y_M & S_M*i^z_M \\
             S_1*j^x_1 & S_1* j^y_1 & S_1* j^z_1 \\
             \dots & \dots & \dots \\
             S_M*j^x_M & S_M*j^y_M & S_M*j^z_M 
        \end{array} 
\right)
       *
        \left( \begin{array}{cccc} 
             P^x_1 & P^x_2 &  \dots & P^x_N \\
             P^y_1 & P^y_2 &  \dots & P^y_N \\
             P^z_1 & P^z_2 &  \dots & P^z_N \
        \end{array} 
        \right)
     = M^i_j * M^P
    \label{TomKanFactMUV}
\end{equation}



We can still
suppose that $M^i_j$ is a square Matrix by padding is with $0$ column or lines.
Padding it with $0$ column correspond to add the projection of a point of coordinate
 $(0,0,0)$, padding with $0$ line correspond to add a camera with scaling factor $0$.
Using a singular value decomposition, $M^u_v$ can be written :


\begin{equation}
   M^u_v=   ^t R_1 \Delta R_2
\end{equation}

With  :

\begin{itemize}
   \item  $R_1 ^t R_1 = Id$  
   \item  $R_2 ^t R_2 = Id$  
   \item  $\Delta$ diagonal matrix.
\end{itemize}

As $M^i_j$ (or $M^p$) is of rank $3$, $M^u_v$ is also of rank $3$.  So  theoretically, $\Delta$  has only $3$
non zero value, due to measurement  error in $u_{n,m},v_{n,m}$ (and also numerical error in computation), the value may be not exactly equal to zero; however,
 $\Delta$  can be best approximated by the $3*3$ matrix corresponding to the highest eigen values (in
absolute values). Suppressing
the corresponding void row of $R_1$ and void column of $R_2$ we obtain a
decomposition of $M^u_v$ in the form :

\begin{equation}
   M^u_v=   r_1  \delta r_2
\end{equation}

With $\delta$ a $3*3$ matrix, $r_1$ a $N*3$ matrix and $r_2$ a $3*N$ matrix, seting arbirtarily 
$r'_2 =  \delta r_2$, we can write  :

\begin{equation}
   M^u_v=   r_1   r'_2
\end{equation}


Also this formula is close to~\ref{TomKanFactMUV}, we cannot identify $M^i_j=r_1$ and $ M^P=r'_2$
because the decomposition is far from unique; in fact for any $3*3$ invertible 
matrix $Q$, we still have :

\begin{equation}
   M^u_v=   (r_1 Q) ( Q^{-1}  r'_2)
\end{equation}

To determine $Q$ we are now using the fact that  the $(i_m,j_m)$ are orthonormal, remember we have : 


\begin{equation}
   M^i_j=   ^t \left( \begin{array}{ccccccc} i_1, i_2 \dots & i_M & j_1 & \dots & j_M 
        \end{array} 
\right)
\end{equation}


We can write the metric constraints :

\begin{itemize}
   \item  $  ^t i_1 Q ^t Q i_1 = 1 $
   \item  $  ^t j_1 Q ^t Q j_1 = 1 $
   \item  $\forall n :  ^t i_n Q ^t Q j_n = 0 $
   \item  $\forall n \neq 1 :  ^t i_n Q ^t Q i_n =  ^t j_n Q ^t Q j_n   $
\end{itemize}

Note that, as there is a global arbirtrary scaling, we make the camera $1$ play a particular
role setting its scaling to $1$, for the other we just impose to have the same scaling
in $x$ and $y$.

As $W=Q ^t Q$ is a symetric matrix, we can estimate $W$  by least mean square using the
above linear equation (there is $6$ unknow on $Q$, we have  $1+2*N$ equations and   $N\geq 3$ ). 
Knowing $W$, it's easy to recover $Q$, we do a singular
value decomposition of $W$ :

\begin{equation}
    W =  ^t R D R
\end{equation}

When all the eigen value are positive, 
\footnote{we see just after when it is not the case}
we can compute a possible value of $Q$ by 

\begin{equation}
    Q  =  ^t R  \sqrt{D}
\end{equation}

This value is not unique, because for any rotation $r$, $Q' = Qr$ will be also a solution. This
non unicity simply reflect the fact that the orientation of the camera can only be computed
up to a global rotation. We will use this fact for processing the case with negative values.

When there exist negative value, the computation is still easy, also we have to use hermitian produce
and hermitian matrices to prove the validity, we can write : 

\begin{equation}
    Q  =  ^t R  \sqrt{|D|} \mathbb{I} = Q' \mathbb{I} 
\end{equation}

Where $\mathbb{I}$  is a diagonal complex matrix containing only $1$ or $i$. We still have :

\begin{equation}
    W= ^t \overline{Q}  Q
\end{equation}

And $\mathbb{I}$ is a hermitian unitary matrix as : 

\begin{equation}
     ^t  \overline{\mathbb{I}}  \mathbb{I} = Id
\end{equation}


As $Q$ is defined up to a global unitary matrix, we can use $Q'$ instead of $Q$.

    % - - - - - - - - - - - - - - - - - - - - - - - - - - - - - - - -
    % - - - - - - - - - - - - - - - - - - - - - - - - - - - - - - - -
    % - - - - - - - - - - - - - - - - - - - - - - - - - - - - - - - -

\section{Triplet selection algorithm}

This section describe the method and code use for selecting triplet, 
as last time I had to go back to it, it seemed to me quite obscure ....
It is implemnted in file {\tt src/uti\_phgrm/cNewO\_OldGenTriplets.cpp}.

Le $S_1$ , $S_2$ be images,

The idea is to select a subset of images, with have a maximum of triple
point, that these point cover the cloud of 

Let :
\begin{itemize}
    \item  $Depth$ = {\tt mHautBase} be the average depth of the scene
    \item  $B_H = \frac{B}{H} $ be base to height ratio
    \item  $B_H^{Lim} = 0.15 $ = {\tt TBSurHLim} be the a constant seting when $\frac{B}{H}$ 
           ratio becomes big;
\end{itemize}

Also the code is obscured by the fact that all computation are done in
integer.

\begin{itemize}
    \item  $Q=30$ = {\tt TQuant} be the constant of quantification.

    \item  $Q_{Bh}=100$ = {\tt TQuantBsH} be the constant of quantification for $\frac{B}{H} $

    \item $Nb_{P1} = 6 $ = {\tt TNbCaseP1} , use for digitalizing the image space

    \item $Nb_{C} = Nb_{P1} * Nb_{P1} $ = {\tt TNbCaseP1},  for efficiency the $2d$ 
          distribution are stored in $1$ board of size $Nb_C$
    
\end{itemize}


The gain when adding a summit is controled by the distance to others,
thi gains is controled by the function :

\begin{itemize}
    \item $B_H(S_1,S_2)  = d(S_1,S_2) / Depth$  

    \item $GB_H(S1,S2) = Q_{Bh} * \frac{B_H}{B_H + B_H^{Lim}}$ 
\end{itemize}







