\section{Relative orientations \er{NO\_AllOri2Im}}
%
The code launches \er{ TestLib NO\_AllOri2Im} (cf. Alg.~\ref{alg:AllOri2Im}) for a pattern of images. Inside the program, it iterates over all images and launches \er{TestLib NO\_Ori2Im} for all connected images (cf. Alg.~\ref{alg:Ori2Im} and Fig.~\ref{fig:workf1Rel}). The main method that calculates the relative orientation between an image pair is detailed in Alg.~\ref{alg:OrInit2Im}.

\vspace{1cm}
\begin{figure}[h!]
\centering
\begin{tikzpicture} 
\node (startImAll) at (0,0) [call] {TestLib NO\_AllOri2Im};
\node (startIm2) at (0,-2)  [call] {TestLib NO\_Ori2Im};
\node (startIm2a) at (-1.5,-1.65) {};
\node (startIm2b) at (-0.75,-1.65) {};
\node (startIm2c) at (0.75,-1.65) {};
\node (startIm2d) at (1.5,-1.65) {};

 
\node (CpleIm) at (5.5,-2) [process] {cNO\_AppliOneCple::CpleIm()};
\node (newNO) at (12.5,-2) [process] {new cNewO\_OrInit2Im(); see Alg.~\ref{alg:OrInit2Im}}; 

\draw [arrow,red] (startImAll) -- node[anchor=west,xshift=-0.5cm] {1} (startIm2a);
\draw [arrow] (startImAll) -- node[anchor=east,xshift=-0.0cm] {2} (startIm2b);
\draw [arrow,blue] (startImAll) -- node[anchor=west,xshift=-0.5cm] {3} (startIm2);
\draw [arrow,green] (startImAll) -- node[anchor=east,xshift=0.5cm] {..} (startIm2c);
\draw [arrow,dashed] (startImAll) -- node[anchor=east,xshift=0.65cm] {N} (startIm2d);

\draw [arrow]  (startIm2) -- (CpleIm) ;
\draw [arrow]  (CpleIm)   -- (newNO)  ;  
\end{tikzpicture}
\caption{The relative orientation workflow. $1$--$N$ signify nodes (i.e. images) in the connectivity graph. Computation is done in parallel.}
\end{figure}\label{fig:workf1Rel}

\subsection{Main classes} 
%
\begin{itemize}
\item cNO\_AppliOneCple -- manager, prepares data for cNewO\_OrInit2Im, contains  the image pair and their names, multiple tie-pts structure, a directory/file manager ...
\item cNewO\_OrInit2Im -- class managing the calculation of a relative orientation 
\item cXml\_O2IComputed -- saves the relative orientation result 
\end{itemize}
 
 
\subsection{Algorithms}
%%%%%%%%%%%%%%%%%%%%%%%%%%%%%%%%%%%%%%%%%% NO\_AllOri2Im
\begin{algorithm}
\caption{\er{ TestLib NO\_AllOri2Im} def in TestAllNewOriImage\_main }
\begin{algorithmic}
\State  $aVIm$ -- vector of images in pattern
\State  $aHom$ -- homologous points\\
\\
\Comment \cmt{Launch rel ori caculation for each image pair}

\For {$aK=0; aK<aVIm \rightarrow size(); aK++$}
 
\State $aVH$ -- vector of overlapping images (with common tie-pts)
\For {$aKH=0$; $aKH<  aVH\rightarrow size(); aKH++$ } 

\State aIm1 = aVIm[aK]
\State aIm2 = aVH[aKH] 

\State \er{TestLib NO\_Ori2Im aIm1 aIm2}
 
\EndFor   
\EndFor  
\end{algorithmic}\label{alg:AllOri2Im}
\end{algorithm}
% 
%%%%%%%%%%%%%%%%%%%%%%%%%%%%%%%%%%%%%%%%%% NO\_Ori2Im
\begin{algorithm}
\caption{\er{ TestLib NO\_Ori2Im} def in TestNewOriImage\_main }
\begin{algorithmic}

%\Comment {fff }
\State  $BenchNewFoncRot()$ 
\\
\\
\Comment \cmt{Preparation of data, i.e. create multiple tie-pts structures, read orientations if InOri }
\State $cNO\_AppliOneCple$ $anAppli(argc,argv)$

\Comment \cmt{Calculate relative orientation }
\State $\textbf{\er{cNewO\_OrInit2Im * aCple = anAppli.CpleIm()}};$

\Comment \cmt{Save result to xml}
\State $cXml\_Ori2Im \&  aXml = {aCple->XmlRes()}$
\State $MakeFileXML(aXml,anAppli.NameXmlOri2Im(true))$
\State $MakeFileXML(aXml,anAppli.NameXmlOri2Im(false));$

\end{algorithmic}\label{alg:Ori2Im}
\end{algorithm}
%
%%%%%%%%%%%%%%%%%%%%%%%%%%%%%%%%%%%%%%%%%% cNewO_OrInit2Im
\begin{algorithm}
\caption{\er{anAppli.CpleIm()} def in NewOri/cNewO\_CpleIm.cpp }
\begin{algorithmic}
\State 
\Comment \cmt{Initialise an object of class $cNewO\_OrInit2Im$}

\State  return new $cNewO_OrInit2Im(mGenOri,mQuick,mIm1,mIm2,\&mMergeStr,mTestSol,mRotInOri,mShow,mHPP,mSelAllCple,*this);  $
 
\end{algorithmic}\label{alg:Ori2ImCont}
\end{algorithm}
%

%%%%%%%%%%%%%%%%%%%%%%%%%%%%%%%%%%%%%%%%%% cNewO_OrInit2Im
\begin{algorithm}
\caption{\er{cNewO\_OrInit2Im} Initial orientation of an image pair}
\begin{algorithmic}
\\
%\Comment {Initialise an object of class $cNewO\_OrInit2Im$}
\State  Creates $PackReduit$ which is a subset of 500 tie-pts chosen heuristically
\State  Save photogrammetric points for the image pair, i.e. HomFloatSym.dat
\State  Initialise the cXml\_Ori2Im for the image pair
\\
\If {$GpsIsInit$}
\State Do something
\EndIf
\\
\State Calculate a homography between 2d points with $cElHomographie::RobustInit$
\State Save results (including residuals, overlap hom, 2d ellipses) to $cXml\_O2IComputed$ class, object $aXCmp$
\\
\\
\Comment \cmt{Test if the scene is not locally planar}
\State ElRotation3D  * aRP = TestOriPlanePatch()
\If {aRP}
\State AmelioreSolLinear(*aRP,)
\EndIf
\\
\Comment \cmt{Test the essential matrix with minimum pts and RANSAC (8pts algo)}
\State ElRotation3D aMRR = TestcRanscMinimMatEss()
\State AmelioreSolLinear(aMRR,);
\\
\\
\Comment \cmt{Test the essential matrix with more pts and RANSAC (8pts algo)}
\State ElRotation3D aRR = RansacMatriceEssentielle()
\\
\\
\Comment \cmt{Test the essential matric - classical test, L1+L2, no RANSAC}
\State ElRotation3D aR =  (aL2 ? mPackPStd.MepRelPhysStd(1.0,true)  : mPackStdRed.MepRelPhysStd(1.0,false)) ;
\\
\\
\Comment \cmt{Test global homography, robust estimation}
\State cResMepRelCoplan aRMC =  ElPackHomologue::MepRelCoplan( .... \\ ...
\State Save the new homography in aXCmp.HomWithR()
\\
\Comment \cmt{Test pure rotation of the camera}
\State cResMepCoc aRCoc= MEPCoCentrik(... \\ ... 
\State Save the rotation in aXCmp.RPure().Ori()
\\
\\
\Comment \cmt{Adjust the solution in a few iterations (until now direct linear solution)}
\State ElRotation3D aSol = aBundle$\rightarrow$OneIterEq(...
\\
\\
\Comment \cmt{Triangulate tie-pts, calculate 3D ellipses, and save}
\State ... aXCmp.Elips() = anElips3D

\\
\State Keep track of computational time with $cXml\_O2ITiming$ \& $aTiming$

\end{algorithmic}\label{alg:OrInit2Im}
\end{algorithm}
%

\begin{algorithm}
\caption{\er{cElRotation3D * TestOriPlanePatch} def in $photogram\/phgr\_mep\_patch\_plane.cpp$}
\begin{algorithmic}
\State to do
\end{algorithmic}\label{alg:TestPlanePatch}
\end{algorithm}
%
\begin{algorithm}
\caption{\er{cElRotation3D * TestcRanscMinimMatEss} def in $photogram\/phgr\_mep\_patch\_plane.cpp$}
\begin{algorithmic}
\State to do
\end{algorithmic}\label{alg:TestMinEss}
\end{algorithm}
%
\begin{algorithm}
\caption{\er{cElRotation3D * RansacMatriceEssentielle} def in $photogram\/phgr\_mep\_patch\_plane.cpp$}
\begin{algorithmic}
\State to do
\end{algorithmic}\label{alg:TestEss}
\end{algorithm}
%
\begin{algorithm}
\caption{\er{cElRotation3D * MEPCoCentrik} def in $photogram\/phgr\_mep\_patch\_plane.cpp$}
\begin{algorithmic}
\State to do
\end{algorithmic}\label{alg:TestRot}
\end{algorithm}