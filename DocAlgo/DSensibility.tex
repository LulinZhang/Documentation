\chapter{Quality control in orientation}

This chapter covers different aspect of quality control of orientation.


    % - - - - - - - - - - - - - - - - - - - - - - - - - - - - - - - -
    % - - - - - - - - - - - - - - - - - - - - - - - - - - - - - - - -
    % - - - - - - - - - - - - - - - - - - - - - - - - - - - - - - - -

\section{Sensibility Analysis : Theoreticall consideration}

\subsection{some tricks}

\subsubsection{tricks}
When $A$ and $B$ are vector, $ \transs A  B$ is as scalar so :


\begin{equation}
     \transs A B =  \transs ( \transs A B) =  \transss B A  \label{Trick:tABEqtBA}
\end{equation}

Then  :

\begin{equation}
     (^t A  B) ^2 =  ^t A B ^t B A  = ^t A (B ^t B) A \label{Trick:tAB2}
\end{equation}

So the term can interpred as the application of the quadratic form $B ^t B$  \footnote{of rank $1$}  to vector $A$.

\subsubsection{tricks}

If $A$ is a symetric positive matrix,
the minimum of quadratic form $F(X) = ^t X A X - 2^t B X$ is reached for $X=A^{-1} B$.
If we write $X' = X + \delta $ :

\begin{equation}
  F(X') -F(X)  = ^t (A^{-1} B+\delta) A (A^{-1} B+\delta) -  2^t B (A^{-1} B + \delta) -F(X)
          =  ^t \delta A \delta 
\end{equation}

Which is always positive as $A$ is positive.

\subsubsection{tricks}
We have also the well known block matrix inverse identity :

\begin{equation}
\left( \begin{array}{cc} 
              A & B \\ 
              C  & D\\ 
        \end{array} 
\right) ^{-1}
= 
\left( \begin{array}{cc} 
              A' & B' \\ 
              C'  & D'\\ 
        \end{array} 
\right) 
= 
\left( \begin{array}{cc} 
              (A-BD^{-1}C)^{-1} & -(A-BD^{-1}C)^{-1} BD^{-1} \\ 
              -D^{-1}C(A-BD^{-1}C)^{-1}  &  D^{-1}+D^{-1}C(A-BD^{-1}C) BD^{-1}\\ 
        \end{array} 
\right) 
\label{Eq:BlockInv}
\end{equation}


    % - - - - - - - - - - - - - - - - - - - - - - - - - - - - - - - -
    % - - - - - - - - - - - - - - - - - - - - - - - - - - - - - - - -
    % - - - - - - - - - - - - - - - - - - - - - - - - - - - - - - - -


\subsection{Least square notation}

Suppose we have $M$ equation of observation with $N$ unknown, $M>N$ :

\begin{equation}
    \sum\limits_{i=1}^N l_i^ m x_i = o_m \; ; \label{Eq:LeastSq:1}
\end{equation}

Noting :
\begin{equation}
    L^m = ^t (l_1^m \;  l_2^m \dots l_N^m)  m \in [1,M] ;  X= ^t (x_1 \; x_2 \dots x_N) 
\end{equation}

Equation~\ref{Eq:LeastSq:1} writes :

\begin{equation}
     ^t L^m  X  = o_m \; ,  m \in [1,M] ;
\end{equation}


As $M>N$ it is generally impossible to annulate all the term , instead we minimise the square of residual $R_2(X)$  :

\begin{equation}
    R^2(X) = \sum\limits_{m=1}^M ( ^tL^m   X - o_m) ^2  
\end{equation}

Using tricks \ref{Trick:tABEqtBA} and \ref{Trick:tAB2} we can write :

\begin{equation}
    R^2(X) = \sum\limits_{m=1}^M  ( (^tL^m   X)^2 - 2 o_m {^tL^m} X + o_m ^ 2) 
           = \sum\limits_{m=1}^M  ( {^t X ({L^m} ^t{L^m}) X} - (2 o_m {^tL^m} )X + o_m ^2)
\end{equation}


Noting the $N \times N$ matrix A, $B$ the $N$ vector and the scalar C :

\begin{equation}
           A = \sum\limits_{m=1}^M { ({L^m} ^t{L^m}) } 
       ;\; B = \sum\limits_{m=1}^M  ( o_m {L^m} ) 
       ;\; C = \sum\limits_{m=1}^M  o_m ^2
 \label{LeastSq:ABC}
\end{equation}

We have :

\begin{equation}
    R^2(X) = ^t X A X - 2^t B X + C
\end{equation}

Obviously $A$ is positive as being the some of squares. The minimum is reached for :

\begin{equation}
     \hat{X}  =  A^{-1} B
\end{equation}

%------------------------ Variance ------------------------------------------

\subsection{Variance}
\label{Sec:VarLsq}

The system being not exactly invertible, for each observation $m$,  
the equation~\ref{Eq:LeastSq:1} is only approximatively satisified by $\hat{X}$,
so we introduce the  residual $\epsilon$  to modelize this uncertainty:

\begin{equation}
     ^tL^m X = o_m + {\epsilon}_m
\end{equation}

To evaluate variance on $X$ we consider  ${\epsilon}_m$  as the realization of random variable. Here I consider that 
each ${\epsilon}_m$ is an indepandant variable, of average $0$ and variance $r_m^2$ \footnote{well  it's probably an heresy
for stasticall point of view to try ro extract information from a single realisation ?} where $r_m^2$ is the empirical residual :


\begin{equation}
      r_m  = ^tL^m \hat{X} - o_m  ; Var({\epsilon}_m) = r_m^2
      \label{Eq:Empir:Var}
\end{equation}

We can then modelize the probabilistic aspect of evaluation of $X$ by the random vector $\tilde X$ :

\begin{equation}
     \tilde X  =  A^{-1}  \sum\limits_{i=1}^M   {L^m}  (o_m + {\epsilon}_m) 
\end{equation}

As $o_m$ is deterministic the variance is :

\begin{equation}
     Var(\tilde X)  =  Var (A^{-1}  \sum\limits_{i=1}^M   {L^m}  {\epsilon}_m)
\end{equation}


Noting the element of $A^{-1}$ :

\begin{equation}
     A^{-1} = ( {a'}_i^j)
\end{equation}

\begin{equation}
     Var( \tilde x_i)  =   Var({\sum\limits_{m=1}^M}  {\sum\limits_{j=1}^N}  {a'}_i^j   {l^m_j}   {\epsilon}_m)
\end{equation}


\begin{equation}
     Var( \tilde x_i)  =   \sum\limits_{m=1}^M  Var({\epsilon}_m)  (\sum\limits_{j=1}^N   {a'}_i^j  {l^m_j}   ) ^2 
\end{equation}



\begin{equation}
     Var(\tilde x_i) = \sum\limits_{m=1}^M (^tL^m \hat{X} - o_m )^2 (\sum\limits_{j=1}^N {a'}_i^j {l^m_j})^2 
\end{equation}

%------------------------ Co-Variance ------------------------------------------

\subsection{Covariance}
\label{Sec:CovLsq}

Similarly, we can compute the covariance :

\begin{equation}
     Cov(\tilde x_i \tilde x_j)  =   \mathbb{E} (
                            ({\sum\limits_{m=1}^M}  {\sum\limits_{k=1}^N}  {a'}_i^k   {l^m_k}   {\epsilon}_m)
                            ({\sum\limits_{n=1}^M}  {\sum\limits_{k=1}^n}  {a'}_i^k   {l^n_k}   {\epsilon}_n)
                       )
\end{equation}

Under the independance hyopthesis, we have


\begin{equation} 
  \forall m,n ,  m\neq n : \mathbb{E} ({\epsilon}_m {\epsilon}_n) = 0 
\end{equation}


\begin{equation}
     Cov(\tilde x_i \tilde x_j)  =     \sum\limits_{m=1}^M  Var({\epsilon}_m) 
                         (\sum\limits_{k=1}^N  {a'}^k_j   {l^m_k}  )
                         (\sum\limits_{k=1}^N  {a'}^k_i   {l^m_k}  )
\end{equation}


%------------------------ Unown substitution ------------------------------------------

\subsection{Unknown elimination}

Computation of variance and co-variance  requires some additional precaution when using
the schurr complement technique as  described in~\ref{UnkAux:Algeb} and~\ref{UnkAux:Var}.
The computation of this paragraph are rather destinated to understant the code modification
impacted by unkown elimination.


We separate the unkown in $X$ and $Y$, where $X$ is the unknown we want to eliminate.

\begin{equation}
     \transK ^m X +  \transL ^m Y  = o_m 
\end{equation}

And the residual  writes : 

\begin{equation}
      R^2 (X,Y) =  \transX (\sum\limits_{m=1}^M  K^m {\transKm} ) X    
                  + 2 (\sum\limits_{m=1}^M  (\transLm Y - o_m  )  \transKm ) X
                  +   \sum\limits_{m=1}^M  (\transLm Y - o_m  ) ^2
\end{equation}

We split $R^2$  in $R^2_y(Y)$ and $r_Y(X)$, where  $R^2_y$ is the part that do not depends of $X$   : 

\begin{equation}
       \Lambda = \sum\limits_{m=1}^M  \KmtKm                       \;\;\;  ; \;
       \Gamma(Y)  = \sum\limits_{m=1}^M  (\transLm Y - o_m  ) K^m 
\end{equation}
\begin{equation}
       r_Y (X) =  ^tX  \Lambda  X    + 2   ^t \Gamma(Y)  X   \;\;\;  ; \;
       R_y^2 (Y) =  \sum\limits_{m=1}^M  (\transLm Y - o_m  ) ^2 
\end{equation}

\begin{equation}
      R^2 (X,Y) =   r_Y(X) +    R_y^2 (Y)  
\end{equation}



To eliminate $X$ in the minimisation, we set $X$  to the value that minimize $r_Y(X)$ for a given  $Y$,
that is :

\begin{equation}
    \hat{X}(Y) = -  \Lambda^{-1}  \Gamma(Y)
    \label{Eq:XHatOfY}
\end{equation}

And the minimum value is :

\begin{equation}
    r_Y(\hat{X}(Y)) 
    = ^t\hat{X}(Y)  \Lambda  \hat{X}(Y)    + 2   ^t \Gamma(Y)  \hat{X}(Y) 
   =   - ^t \Gamma(Y) \Lambda^{-1} \Gamma(Y)
\end{equation}


In unkown elimination we suppose that $X=\hat{X}(Y)$ and compute only : 

\begin{equation}
      \breve{R}^2 (Y) = R^2(\hat{X}(Y),Y)  =     R_y^2 (Y)   -  ^t \Gamma(Y) \Lambda^{-1} \Gamma(Y)
\end{equation}

We develop :

\begin{equation}    \breve{R}^2 (Y) =
                       \sum\limits_{m=1}^M  (\trans L^m Y - o_m  ) ^2
                  -    (\sum\limits_{m=1}^M  (\trans L^m Y - o_m  )  \trans K^m )  
                       \Lambda^{-1}
                       (\sum\limits_{m=1}^M  (\trans L^m Y - o_m  )  K^m ) 
\end{equation}

Noting :


\begin{equation}  
     \breve{A} =  \sum\limits_{m=1}^M{L^m}{\trans L^m}
               -(\sum\limits_{m=1}^M {L^m} \trans {K^m})   \Lambda^{-1}  (\sum\limits_{m=1}^M   {K^m} \trans {L^m} ) 
\end{equation}  

\begin{equation}  
     \breve{B}  =   \sum\limits_{m=1}^M (o_m L^m ) 
                - (\sum\limits_{m=1}^M  {L^m}  \trans K^m) \Lambda^{-1}  (\sum\limits_{m=1}^M o_m K^m )
\label{Var:BChap}
\end{equation}  

We have :

\begin{equation}  
      \breve{R}^2 (Y) = \trans Y \breve{A} Y - 2  \trans \breve{B}  Y + Cste
\end{equation}


And the estimation of $\breve{Y}$ of $Y$ by least square : 


\begin{equation}  
      \breve{Y} =  \breve{A}^{-1}  \breve{B}
\end{equation}

We write :

\begin{equation}  
     \Theta =  \sum\limits_{m=1}^M {L^m} \trans {K^m}   
\end{equation}  


\begin{equation}  
     \breve{A} =  \sum\limits_{m=1}^M{L^m}{\trans {L^m}} - \Theta \Lambda^{-1}   {\trans} \Theta 
     \label{FinalBreveA}
\end{equation}  

\begin{equation}  
     \breve{B}  =   \sum\limits_{m=1}^M (o_m L^m ) 
                - \Theta \Lambda^{-1}  (\sum\limits_{m=1}^M o_m K^m )
               = \sum\limits_{m=1}^M  o_m (L^m -\Theta \Lambda^{-1} K^m)
     \label{FinalBreveB}
\end{equation}  

Defining $\breve{L}^m$, we have :

\begin{equation}  
     \breve{L}^m   =   (L^m  - \Theta  \Lambda^{-1} K^m)
\end{equation}  

\begin{equation}  
     \breve{B}  = \sum\limits_{m=1}^M  o_m \breve{L}^m
\end{equation}  


Using $\breve{A}$, $\breve{B}$  and $\breve{L}^m$ we finaly can compute the variance and
covariance of $\breve Y$ with formula equivalent to~\ref{Sec:VarLsq} and~\ref{Sec:CovLsq}~.
We consider the random vector $\tilde Y$ :

\begin{equation}  
     \tilde{Y}  =   \breve{A}^{-1}  \sum\limits_{m=1}^M  (o_m + \epsilon _m) \breve{L}^m
\end{equation}  

We have :

\begin{equation}
     Var(\tilde{y}_i) =  \sum\limits_{m=1}^M  Var(\epsilon _m)  (\sum\limits_{j=1}^N {\breve {a}'}{_i^j}  {\breve l}{^m_j})^2 
\label{Final:Var}
\end{equation}


\begin{equation}
     Cov(\tilde y_i \tilde y_j)  =     \sum\limits_{m=1}^M  Var({\epsilon}_m) 
                         (\sum\limits_{k=1}^N  {\breve {a'}}^k_j   {\breve {l}}^m_k  )
                         (\sum\limits_{k=1}^N  {\breve {a'}}^k_i   {\breve {l}}^m_k  )
\end{equation}

\subsection{Practicle aspects on unknown elimination in MicMac}

Practically, in MicMac, the unkwon elimination is essentially used  to eliminate, for each tie points, 
the $3$d point that project on each image. This is done "\`a la vol\'ee" (on the flight ?) with
the following procedure, for each tie point :

\begin{itemize}
    \item the unknown of the $3d$ point are always located to the same place (say they are unknown $1,2,3$)
    \item the observation are used in accumulator matrix $A,B,C$ using equation \ref{LeastSq:ABC} (ignoring
          for now the future elimination);
    \item  then $\Lambda$ and $\Theta$ are computed and equations \ref{FinalBreveA} and \ref{FinalBreveB} to
           modify the accumulator $A,B,C$
    \item the part of the accumulator $A,B,C$  corresponding to unknow $[1-3]$ are reseted;
\end{itemize}

So at the end, the accumulator $A$ and $B$ contains the  global $\breve A$  and $\breve B$ and allow to
compute the optimal value $\breve Y$ . For variance-covariance,
this way of proceeding ,  raise a probleme for computing the residual for estimation of $Var(\epsilon _m)$  that
require the value of unknowns as shown in \ref{Eq:Empir:Var}, we know the value for $Y$, but not for $X$.
There is two possibility :

\begin{itemize}
   \item  use  formula \ref{Eq:XHatOfY} , with  $\breve Y$  as value, this create an iteration offset;
   \item use the value computed from bundle intersection, which is also an approximation.
\end{itemize}

For now the second solution is used .



    % - - - - - - - - - - - - - - - - - - - - - - - - - - - - - - - -
    % - - - - - - - - - - - - - - - - - - - - - - - - - - - - - - - -
    % - - - - - - - - - - - - - - - - - - - - - - - - - - - - - - - -

\subsection{Sensibility}


\begin{equation}
F(X) = ^t X M X = F(y,Z)= 
\left( \begin{array}{cc} 
              y &  ^t Z \\ 
        \end{array} 
\right)
\left( \begin{array}{cc} 
              a & B \\ 
              ^t B  & D\\ 
        \end{array} 
\right)
\left( \begin{array}{c} 
              y \\ 
              Z \\ 
        \end{array} 
\right)
= a y^2 + 2y^t B Z + ^t Z D Z
\label{EqSensib1}
\end{equation}

For a given $y$, $F(y,Z)$ is minimal for :

\begin{equation}
Z_{min}(y) = -y  D^{-1} B
\end{equation}

And the minimal value is :

\begin{equation}
V_{min}(y) = F(y,Z_{min}(y)) =  y^2 (a- ^t B D^{-1} B)
\end{equation}


In our  case where $A=a$ is a $1$ dimensionnal (scalar) we can then write :

\begin{equation}
V_{min}(y) =   y^2 (a- ^t B D^{-1} B) = \frac{x^2}{a'}
\end{equation}

So using equation ~\ref{Eq:BlockInv}, $V_{min}(x)$ can be easily computed  from the inverse matric. If
we have a "bad" value of $y$, we have two estimation of the impact
on  $F$ :

\begin{itemize}
   \item a "pessimistic" $a y^2$;
   \item a "optimistic" $\frac{y^2}{a'}$.
\end{itemize}

So know, if we explain the empirical least square error $R$, by
a bad estimation on $y$, we have two estimation of the sensibility/accuracy of $y$ ,
optimisitic in \ref{Sens:Optim} and pessimistic in \ref{Pess}:

\begin{equation}
  \sqrt{\frac{a}{R}}
  \label{Sens:Optim}
\end{equation}


\begin{equation}
  \sqrt{a'R}
  \label{Sens:Pess}
\end{equation}


    % - - - - - - - - - - - - - - - - - - - - - - - - - - - - - - - -
    % - - - - - - - - - - - - - - - - - - - - - - - - - - - - - - - -
    % - - - - - - - - - - - - - - - - - - - - - - - - - - - - - - - -

\section{Sensibility Analysis : Use in MicMac}

The computation of these different value can be done in {\tt Martini} command,
by setting to true the optional parameter {\tt ExportSensib}. Different
value are exported at the end of computation; all the file are located
in the same folder containing the orientation generated and they  have name
begining  {\tt Sensib}.


There is four matrix file, exported as images in float format. This files
are : 

\begin{itemize}
     \item {\tt Sensib-MatriceCov.tif}  contain the covariance matrix,
           this is the matrix resulting from unknown elimination (this of \ref{FinalBreveA});

     \item {\tt Sensib-MatriceCorrelDir.tif}  contain the correlation extracted from
           direct covariance matrice i.e.  $ \frac{a^i_j}{\sqrt{a^i_i * a^j_j}}$;

     \item {\tt Sensib-MatriceCorrelInv.tif}  contain the correlation extracted from
           invert covariance matrice $\breve{A} {^{-1}}$  ;
\end{itemize}

Probably the {\tt Sensib-MatriceCorrelDir.tif} is what  is most currently used
and known as correlation matrices. When exploring the images with the {\tt Vino}
tool, the value are printed using short names, for example when grabing the 
window, one can get the following messages :


\begin{itemize}
     \item {\tt V=0.452} and {\tt [Ima4:cZ] [Ima3:T12] (P=41,32)}
     \item this mean than the correlation is $0.452$ between $Z$ coordinate of center
           image $4$  and $\theta_{12}$ of image $3$;
      \item as short name are used for variables in {\tt Vino} , a file containing conversion
            between short and long name is generated.
\end{itemize}


An example of conversion file :

\begin{verbatim}
##############  Intrinseque Calibration Correspondance ##############
 Cal0 => ./Ori-AllRel/AutoCal_Foc-24000_Cam-PENTAX_K5.xml
##############  Extrinseque Calibration Correspondance ##############
 Ima0 => IMGP7029.JPG
 Ima1 => IMGP7030.JPG
...
\end{verbatim}

The file {\tt Sensib-Data.xml} contains information on  variance, uncertainty  ...
regarding individual variable . Three value are given, corresponding
to different formula  :

\begin{itemize}
    \item formula \ref{Sens:Optim}  correspond to {\tt <SensibParamDir>};
    \item formula \ref{Sens:Pess}   correspond to {\tt <SensibParamInv>};
    \item formula \ref{Final:Var}   correspond to {\tt <SensibParamVar>}.
\end{itemize}

Here is an example with an acquisition mixing GPS and photogrammetry. Probably the
 {\tt <SensibParamVar>} is the more realistic evaluation of uncertainty.

\begin{verbatim}
     <SensibDateOneInc>
          <NameBloc>cBaseGPS</NameBloc>
          <NameInc>x</NameInc>
          <SensibParamDir>0.0985765205323673732</SensibParamDir>
          <SensibParamInv>0.577266441576786526</SensibParamInv>
          <SensibParamVar>0.00228406097850316443</SensibParamVar>
     </SensibDateOneInc>
...
     <SensibDateOneInc>
          <NameBloc>Cal0</NameBloc>
          <NameInc>F</NameInc>
...       <SensibParamVar>15.0011827674349423</SensibParamVar>
     </SensibDateOneInc>
     <SensibDateOneInc>
          <NameBloc>Cal0</NameBloc>
          <NameInc>PPx</NameInc>
...       <SensibParamVar>1.68730018610615495</SensibParamVar>
     </SensibDateOneInc>
...
     <SensibDateOneInc>
          <NameBloc>Ima0</NameBloc>
          <NameInc>Cx</NameInc>
          <SensibParamDir>0.10381674122562666</SensibParamDir>
          <SensibParamInv>1.08339374408743527</SensibParamInv>
          <SensibParamVar>0.00384393697511320725</SensibParamVar>
     </SensibDateOneInc>
...
\end{verbatim}

%-------------------------------------------------------------------
%-------------------------------------------------------------------
%-------------------------------------------------------------------

\section{Spatial repartition of residual}

This option is avalaible with {\tt ExpImRes} option of {\tt Campari}.
It generates images representing the spatial repartition of residual
in the sensor plane \footnote{extension may come in the ground plane and in $3$d} . 
These images are located in the subfolder
{\tt ImResidu} of orientation folder. It can generate :

\begin{itemize}
   \item images  representing the module of residual 
          (name of generated images will begin by {\tt ResAbs-});

   \item images  representing the signed residual;
          (name of generated images will begin by {\tt ResSign-});

   \item images  representing the vector $X,Y$ of residual according to some
         mathematical modelization that will be described later 
         (name of generated images will begin by {\tt ResX-} and {\tt Res-Y});

   \item images  representing the total weight accumulated in each pixel
          (name of generated images will begin by {\tt RawWeight-});
          this image can be helpfull to know were previous images are reliable;
   

\end{itemize}

These images can be generated for  different configuration :

\begin{itemize}
   \item one image for each internal calibration, name of these image contains {\tt Cam-},
         for example {\tt  RawWeight-Cam-\dots};

   \item one image for each internal pose, name of these image contains {\tt Cam-},
         for example {\tt  RawWeight-Pose-\dots};

   \item one image for each internal pair of overlapping, name of these image contains {\tt Pair-},
         for example {\tt  RawWeight-Pose-\dots};
\end{itemize}

For the pair, only the signed residual and weighting are generated (because others
would have no meaning).

The $X,Y$ residual is computed this way \dots 

%-------------------------------------------------------------------
%-------------------------------------------------------------------
%-------------------------------------------------------------------

\section{Detecting fault in GCP and  Robust "Bascule"  with {\tt BAR}}
\label{Sec:BAR}

     % - - - - - - - - - - - - - - - - - - - - - - - - -
\subsection{When is it usefull}

The command {\tt GCPBascule} and {\tt GCPCtrl} do the assumption that the 
measures comming for {\tt GCP} contain no gross error (outlayers) .
This is generally a reasonnable assumption as measurement comes from human seizing and they contain only
gaussian error adapted to least square fitting. However in "real life", sooner or later, will occur case
where this data may contain gross error :

\begin{itemize}
  \item error in GCP file like naming convention;
  \item error in point seizing when associating a point to its image projection;
  \item unvolonter move a point after seizing it;
  \item  \dots
\end{itemize}

The command {\tt BAR} compute a "robust" bascule  that is expected to be resistant
to (a reasonnable amount of) outlayers. More important, it provide a detailled
diagnostic that may help to detect the oulayer both in GCP files and in images
measurement.

     % - - - - - - - - - - - - - - - - - - - - - - - - -

\subsection{Mathematicall modeling}

The  parameter of the bascule (Hemert transform) is computed using Ransac. To generate
a solution :

\begin{itemize}
   \item we select $3$ random GCP;
   \item for each selected GCP we select $2$ random image where the point is measured;
   \item then we compute by bundle intersection the coordinat of the point in the relative system;
   \item finaly having a slightly  redundant system ($9$ observation for $7$ degrees of freedom), we
         compute by least square a solution;
\end{itemize}

The score to select a solution $S$ is the sum of the \emph{modified} reprojection
errors between all $S(G_k)$ in all the images where it is measured  . 
This  \emph{modified} projection take to parameter $D_0$ and $\gamma$, let $D$ being the
standard reprojection error, we use the formula :

\begin{equation}
    (\frac{D_0 D}{D_0 + D}) ^ \gamma \label{Eq:Reproj:Bar}
\end{equation}

Where :

\begin{itemize}
   \item $D_0$ limit the impact of gross errors;
   \item the choice of the exponent $ \gamma$ is typycal of $L_1$ solution with  $ \gamma=1$,
         least square with  $ \gamma=2$ \dots
\end{itemize}


     % - - - - - - - - - - - - - - - - - - - - - - - - -
\subsection{Syntax}

The syntax :

\begin{verbatim}
 mm3d BAR 
*****************************
*  Help for Elise Arg main  *
*****************************
Mandatory unnamed args : 
  * string :: {Pattern of images}
  * string :: {Orientation}
  * string :: {Name of 3D Point}
  * string :: {Name of 2D Points}
Named args : 
  * [Name=NbRan] INT :: {Number of random, Def=30000}
  * [Name=SIFET] bool :: {Special export for SIFET benchmark}
  * [Name=Expos] REAL :: {Exposure for dist, 1->L1, 2->L2 , def=1}
  * [Name=Out] string :: {Export computed orientation, Def=BAR+${-Orientation}}
  * [Name=RatioAlertB] REAL :: {Ratio of alert for bundle reproj (Def=3)}
  * [Name=MaxEr] REAL :: {Typical max error image reprojecstion,Def=10}
\end{verbatim}


The four mandatory parameters are the same as {\tt GCPCtrl}. The {\tt Expos} parameter
correspond to $\gamma$ in equation~\ref{Eq:Reproj:Bar}, the {\tt MaxEr}
correspond to $D_0$ in equation ~\ref{Eq:Reproj:Bar}.

     % - - - - - - - - - - - - - - - - - - - - - - - - -
\subsection{Analyse of results in file {\tt ResulBar.txt}}

The command {\tt BAR} generate new orientation, but probably the most interesting result 
is the file {\tt ResulBar.txt} as  the objective of photogrammetry is not to get the "less bad" orientation 
using robust method, but to get the best one by supressing the error.

As the error can have two origin (GCP or image measurement), the file  {\tt ResulBar.txt} contains
two kind of reprojection error :

\begin{itemize}
   \item the projection of GCP on image (named {\tt ReprojTer}) , this errors correspond to a mix of
         GCP errors and  image errors;

   \item the projection of a point computed from bundle intersection using Ransac method (named {\tt ReprojBundle}); 
         this error is independant of any GCP measurement and correspond to "quality" of the work
          of the operator seizing the points.
\end{itemize}

The file contains three parts :

\begin{itemize}
   \item One part for global statistic;

   \item One part for statistic on each GCP;

   \item One part contains for each GCP and each images where it is seized, the "terrain" and "bundle" 
         reprojection error;
\end{itemize}


We illustrate with a "real" file  where several gross error have been artificially added.
Here is an example of global parts :


\begin{verbatim}
================== Global Stat  ==========================
================== Global Stat  ==========================
   Aver Dif    Bundle/Ter (0.000744 0.001767 0.198431 )
   Aver AbsDif Bundle/Ter (0.001341 0.002444 0.242504 )
   Worst reproj Ter   1979.695852 for pt 621 on image IMG_1281.JPG
   Worst reproj Bundle 1000.487300 for pt 111 on image IMG_1170.JPG
\end{verbatim}

The first two line are the average difference, and the average absolute difference between
bundle point after bascule, and GCP point. Then follow the worst value of reprojection
for terrain an bundle. The ide is  that if these two value are low, there is probably
no gross error.

The stat by points looks like :


\begin{verbatim}
   ----------------------------------------------
[NamePt] :
   Bundle D=Dist(GCP,Bundle) : GCP-Bundle
   Worst ReprojTer ReprojBundle  ImWorsTer ImWorstBundle
   Aver   ReprojTer ReprojBundle  on NbIma
    ----------------------------------------------
[111] :
   Bundle, D=0.000845 : (-0.000624 -0.000561 -0.000100)
   Worst 1000.538  1000.487  IMG_1170.JPG  IMG_1170.JPG
   Aver  063.262  067.043 on 16
[112] :
   Bundle, D=0.000679 : (-0.000123 0.000177 0.000644)
   Worst 000.817  000.566  IMG_1163.JPG  IMG_1163.JPG
   Aver  000.521  000.286 on 19
...
[623] :
   Bundle, D=1.899735 : (0.001914 0.005929 1.899725)
   Worst 1370.917  000.700  IMG_1304.JPG  IMG_1153.JPG
   Aver  1227.879  000.316 on 38
\end{verbatim}


On this data, it can expected that :

\begin{itemize}
   \item with point {\tt [111]}, the GCP is Ok (low distance betwen bundle intersection an GCP),
         but there is error in image measuremnt : high value in bundle reprojection;

   \item with point {\tt [112]}, GCP and imahe measurement are Ok;

   \item with point {\tt [623]}, the image measurement is OK, but the GGP is not (or the orientation
         has diverged on this part).
\end{itemize}

Here is an extract from the detailled part :

\begin{verbatim}
================== Detail per point & image ==========================
 ================== Detail per point & image ==========================

    ----------------------------------------------
[NamePt]
    | ReprojTer | ReprojBundle | Image
    | ReprojTer | ReprojBundle | Image
    ....
    ----------------------------------------------
[111]
  #@+ | 1000.538 | 1000.487 | IMG_1170.JPG
      | 001.243 | 000.542 | IMG_1171.JPG
      | 001.085 | 000.362 | IMG_1173.JPG
      | 001.205 | 000.877 | IMG_1175.JPG
...
[112]
      | 000.444 | 000.363 | IMG_1170.JPG
....
  #@  | 000.817 | 000.566 | IMG_1163.JPG
[511]
  #   | 095.410 | 000.619 | IMG_1251.JPG
....
   @+ | 065.827 | 001.166 | IMG_1261.JPG
    + | 061.008 | 001.114 | IMG_1262.JPG

[623]
...
  #   | 1370.917 | 000.256 | IMG_1304.JPG
...
   @  | 1173.665 | 000.700 | IMG_1153.JPG
...
\end{verbatim}

A few comment :

\begin{itemize}
   \item on point  {\tt [111]}, it can been seen that the error is located on image {\tt IMG\_1170.JPG}  
         in the first column the {\tt \#@+ } means :
    \begin{itemize}
         \item {\bf \tt \#} : the images maximizes terrain reprojection for given point;
         \item {\bf \tt @} : the images maximizes bundle reprojection for given point;
         \item {\bf \tt +} : the bundle error is over $R$ time the median error of bundle error,
              $R$ is set by {\tt RatioAlertB}  (def value$=3$);
    \end{itemize}
   \item on point  {\tt [112]}, no problem appears;
   \item on point  {\tt [511]}, several point are marked  {\bf \tt +};
   \item point  {\tt [623]}, the detailled results confirm that there is probably a problem between  {\tt GCP}
         coordinates and bundle,  but that image seizing are coherents;
\end{itemize}

It is hoped that this tool is sufficient to detect the most probable errors. It does
not aim to detect all the errors, the standard recommanded method being, detect
the biggest error, suppress it and iterate as long as there exist gross error.






